% -----------------------------------------------------
\chapter{Java}
% -----------------------------------------------------

Let's dissect the following code block that sums two numbers.

\begin{java}
import java.util.*;
public class Solution {
    public static void main(String[] args) {
        Scanner in = new Scanner(System.in);
        int a, b;
        a = in.nextInt();
        b = in.nextInt();
        System.out.println(a + b);
    }
}
\end{java}

\begin{quest}
\item
	What does \texttt{String[] args} mean in the hello world program?

\begin{ans} \texttt{args} is an array of strings. \end{ans}
\end{quest}

\paragraph*{Arrays}
\begin{quest}
\item Declare an array of strings.
\begin{ans}
\begin{java}
String[] arr;
\end{java}
\end{ans}

\item Initialize an array of integers containing 1, and 2.
\begin{ans}
\begin{java}
int[] arr = {1, 2};
\end{java}
\end{ans}

\item Initialize an array of strings containing "BMW" and "Ford".

\begin{ans}
\begin{java}
String[] carBrands = {"BMW", "Ford"};
\end{java}
\end{ans}

\item Given an array,
\code{String[] carBrands = \{"BMW", "Ford"\}; }
print the first element.

\begin{ans}
\begin{java}
System.out.println(carBrands[0]);
\end{java}
\end{ans}
\end{quest}


\begin{quest}
\item
	\texttt{in.nextInt()} ?

\begin{ans}
content
\end{ans}

\end{quest}




It may be a good idea to go through all of this: \url{https://www.w3schools.com/java/default.asp}


\section{Gradle}

\url{https://youtu.be/aYu994I8Z6I?list=PLMxpKvJf0K0QUyvmkKZu7WpwTVzdePGTl}

\begin{itemize}
\item Gradle is an pen-source build atomation tool.
\item Gradle is the official android build tool; maintained by Android SDK Tools team
\item Gradle build scripts are written using Groovy
\end{itemize}


\paragraph*{WHy Gradle?}
\begin{itemize}
\item highly customizable and extensible
\item used for multiple languages
\item it's fast: reuses outputs from previous executions, processing only inputs that changed; executes tasks in parallel
\item higher performance than its peers (such as Maven)

\end{itemize}

\paragraph*{Gradle project tasks}



