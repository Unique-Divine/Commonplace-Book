\chapter{Quotes}

\begin{quotation}
“your mind is for having ideas, not holding them.” - David Allen
\end{quotation}

\begin{quotation}
``Some very specific but seemingly mundane behaviors, when applied, produce the capacity to exist in a kind of sophisticated spontaneity, which, in my experience, is a key element to a successful life.'' - David Allen
\end{quotation}

\begin{quotation}
“We should hunt out the helpful pieces of teaching and the spirited and noble-minded sayings which are capable of immediate practical application–not far far-fetched or archaic expressions or extravagant metaphors and figures of speech–and learn them so well that words become works.” - Seneca
\end{quotation}

\begin{quotation}
``I have a friend who’s an artist and has sometimes taken a view which I don’t agree with very well. He’ll hold up a flower and say “look how beautiful it is,” and I’ll agree. Then he says “I as an artist can see how beautiful this is but you as a scientist take this all apart and it becomes a dull thing,” and I think that he’s kind of nutty. First of all, the beauty that he sees is available to other people and to me too, I believe…

I can appreciate the beauty of a flower. At the same time, I see much more about the flower than he sees. I could imagine the cells in there, the complicated actions inside, which also have a beauty. I mean it’s not just beauty at this dimension, at one centimeter; there’s also beauty at smaller dimensions, the inner structure, also the processes. The fact that the colors in the flower evolved in order to attract insects to pollinate it is interesting; it means that insects can see the color. It adds a question: does this aesthetic sense also exist in the lower forms? Why is it aesthetic? All kinds of interesting questions which the science knowledge only adds to the excitement, the mystery and the awe of a flower. It only adds. I don’t understand how it subtracts.'' - Richard Feynman on the beauty of a flower
\end{quotation}

\begin{quote}
``AI began with an ancient wish to forge the gods.'' - Pamela McCorduck, Machines Who Think, 1979
\end{quote}

\begin{quote}
``We cannot solve our problems with the same thinking we used when we created them.'' - Albert Einstein
\end{quote}

\begin{quote}
	``I can calculate the motion of heavenly bodies, but not the madness of people.'' - Isaac Newton
\end{quote}

