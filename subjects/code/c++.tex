
% -----------------------------------------------------
\chapter{C++}
% -----------------------------------------------------

C++ source code files end with a .cpp extension.

Hello world program: Run these and find out which one works.

\begin{cpp}
#include <iostream>

int main()
{
    std::cout << "Hello, world!";
    return 0;
}
\end{cpp}

Compiling and executing the C++ program:
\begin{enumerate}
\item
Step 1 is to install the gcc compiler.

\code{hi}


\item
Verify the install of g++ and gdb with
\code{whereis g++} and \code{whereis gdb}

To install gdb (linux or WSL), use \code{sudo apt-get install build-essential gdb}

\end{enumerate}


\paragraph*{\href{https://www.openmp.org//wp-content/uploads/openmp-examples-4.5.0.pdf}{Open MP}}

To import:
\begin{cpp}
#include <omp.h>
\end{cpp}

People use OpenMP for shared memory parallelization.


\subsection*{Header files}

C++ programs consist of more than just .cpp files. They also use **header files**, which can have a .h extension, .hpp extension, or even none at all.

\begin{quest}
\item What is a \texttt{.h} file?
\begin{ans}
header file
\end{ans}


\item What is the purpose of a header file?
\begin{ans}
Header files allow us to put declarations in one location and then import them wherever we need them. This can save a lot of typing in multi-file programs.
\end{ans}

\item What's contained in a \texttt{.h} file?
\begin{ans}
..
\end{ans}
\end{quest}


\subsection*{References \& Further Reading}
\begin{itemize}
\item
	\href{https://www.tutorialspoint.com/cprogramming/c_header_files.htm}{C header files}
\item
	\href{https://www.learncpp.com/cpp-tutorial/header-files/}{learncpp.com/.../header-files}
\end{itemize}