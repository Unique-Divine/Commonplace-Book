\chapter{Math \& Code}


%%%%%%%%%%%%%%%%%%%%%%
\section{Hugo Web Design}
%%%%%%%%%%%%%%%%%%%%%%

Start Sep 16 (Mike Dane Tutorial Series)

\subsection{Intro to Hugo -\href{https://youtu.be/qtIqKaDlqXo}{(video)} }
\begin{itemize}
	\item
	Hugo is a static site generator.
	\item
	Static website generators allow you to compromise between writing a bunch of static html pages and using a heavy, and potentially expensive, content management system.
	\item
	Why Hugo? It's extremely fast.
	\item
	2 kinds of websites, dynamic and static. Dyanmic ex. Facebook. Facebook pages are dynamically generated for each user. For static websites, what you see is what you get.
	\item
	Static websites are notoriously harder to maintain b/c you lack some of the flexibility of things  on a dynamic site. Usually you can't use much conditional logic, functions, or variables.
	\item
	However, static pages are extremely fast.
	\item
	Hugo is great for a blog, portfolio website, etc.
	\item
	Hugo doesn't explicity require you to write a single line of HTML code.
	\item
	Flexibility | With that said, if you want to go in and change every little detail of the layout of the site, you cna do that. You can write as much of the HTML and have as much control as you'd like.
	\item
	Hugo is 100\% free and open-source.
\end{itemize}

\subsection{Intalling Hugo on Windows - \href{https://youtu.be/G7umPCU-8xc?list=PLLAZ4kZ9dFpOnyRlyS-liKL5ReHDcj4G3}{(video)} }
\begin{itemize}
	\item
	Mine's already installed. I'll skip this for now.
	\item
	a
	\item
	a
\end{itemize}

\subsection{Creating a new site - \href{https://youtu.be/sB0HLHjgQ7E?list=PLLAZ4kZ9dFpOnyRlyS-liKL5ReHDcj4G3}{(video)} }
\begin{itemize}
	\item skip, a bit too easy

\end{itemize}

\subsection{Installing \& Using Themes - \href{https://youtu.be/L34JL_3Jkyc?list=PLLAZ4kZ9dFpOnyRlyS-liKL5ReHDcj4G3}{(video)} }
\begin{itemize}
	\item my themes are already installed
\end{itemize}

\subsection{Creating \& Organizing Content - \href{https://youtu.be/0GZxidrlaRM?list=PLLAZ4kZ9dFpOnyRlyS-liKL5ReHDcj4G3}{(video)} }
\begin{itemize}
	\item
	Hugo has 2 types of content: single pages and list pages
	\item
	List content lists other content on the site. You can call this a list page.
	\item
	Individual blog posts are single pages.
	\item
	Your posts should not just be in the content directory. They should be in directories inside the content directory.
	\item
	A list page is automatically created for directories inside the content folder.  Hugo automatically does this. Note, this only occurs for directories at the ``root" level of the content directory. For example:  \texttt{content/post/} would generate a list page at \texttt{site.com/post/}, but \texttt{content/post/dir0} would not.
	\item
	If you want a list page to be generated for a dir that is not at the root level of the content dir, you have to create an \textbf{index filed}, \texttt{\_index.md}. For a convenient and efficient way to do this from the cmd, use \texttt{hugo new post/dir0/\_index.md} (above above example), then there will be a list page for dir0. 	Content can also be added to\texttt{\_index.md} and it should show up on the page.
	\item
	Additionally, for list pages that are aautomatically generate by hugo, you can edit the content by adding an index.md to those as well. Ex. \texttt{~content/post/\_index.md}.
\end{itemize}

\subsection{Front Matter- \href{https://youtu.be/Yh2xKRJGff4?list=PLLAZ4kZ9dFpOnyRlyS-liKL5ReHDcj4G3 }{(video)} }
\begin{itemize}
	\item
	Front matter in Hugo is what is commonly called meta data.
	\item
	Front matter is data about our content files.
	\item
	The metadata automatically generated by Hugo at the top of md files when using \texttt{hugo new } is front matter
	\item
	Front matter is stored in key-value pairs
	\item
	Front matter can be written in 3 different languages: YAML, TOML, and JSON
	\item
	The defualt lang for front matter in Hugo is YAML
	\item
	YAML - indicated by ``--'',
	\item
	TOML - indicated by "+++" and uses "=" instead of ":",
	\item
	JSON - indicated
	\item
	You can create your own custom front matter variables.
	\item
	Front matter is super powerful in its utility.
\end{itemize}


\subsection{Archetypes - \href{https://youtu.be/bcme8AzVh6o?list=PLLAZ4kZ9dFpOnyRlyS-liKL5ReHDcj4G3 }{(video)} }
\begin{itemize}
	\item
	How does the default front matter from using \texttt{hugo new ~.md} get selected? Short answer: archetypes
	\item
	An archetype is basically the default front matter template for when you create a new content file.
	\item
	Archetypes are modified under \texttt{static/themes/archetypes/default.md}
	\item
	Suppose your content dir has a subdirectory, \texttt{content/dir0}. If you wanted to create an archetype for the files in dir0, you'd simply create \texttt{dir0.md} inside the archetypes dir.
\end{itemize}

\subsection{Shortcodes - \href{https://youtu.be/2xkNJL4gJ9E?list=PLLAZ4kZ9dFpOnyRlyS-liKL5ReHDcj4G3 }{(video)} }
\begin{itemize}
	\item
	Shortcodes are predefined chunks of HTML that you can insert into your markdown files.
	\item
	Let's say you have a md file that you want to spice up by adding in some custom HTML. For instance, maybe you'd like to embed a YouTube video. Normally this would require lots of HTML that you'd have to paste it. Shortcodes can allow you to sidestep this. Hugo comes with a YouTube video shortcode predefined.
	\item
	General shortcode syntax \texttt{}
	\item
	Youtube shortcode | For a YouTube video with url, ``youtube.com/watch?v=random-text", the shortcode we'd use to embed would be \texttt{} because ``random-text'' is the id of the youtube video and the only parameter for that shortcode.
\end{itemize}

\subsection{Taxonomies - \href{https://youtu.be/pCPCQgqC8RA?list=PLLAZ4kZ9dFpOnyRlyS-liKL5ReHDcj4G3 }{(video)}}
\begin{itemize}
	\item
	Taxonomies in hugo are basically ways that you can logically group different pieces of content together in order to organize it in a more efficient way.
	\item
	Hugo provids 2 defualt taxonomies: tags \& categories
	\item
	All taxonomy information is declared in front matter.  In YAML, tags has the syntax \texttt{tags: ["tag0", "tag1", $\ldots$]}
\end{itemize}

\subsection{Templates - \href{https://youtu.be/gnJbPO-GFIw}{(video)}}
\begin{itemize}
	\item
	Templates here mostly refers to HTML templates. If you're not comfortable writing HTML, CSS, and coding for the web, templates might be a little bit above your head.
	\item
	A hugo theme is actually made up of hugo templates.
	\item
	Any template that you use in Hugo is going to be inside \texttt{themes/theme-name/layouts}. This is where all the templates live.
	\item
	\texttt{~/layouts/default} usually contains a default style for list and single pages by use of \texttt{list.html} and \texttt{single.html}.
\end{itemize}

\subsection{List Templates - \href{https://youtu.be/8b2YTSMdMps}{(video)}}
\begin{itemize}
	\item
	List templates give default HTML layout to list content files.
	\item

\end{itemize}

\subsection*{Resources}
\begin{itemize}
	\item
	\href{https://www.linkedin.com/learning/learning-static-site-building-with-hugo-2/build-a-static-site-with-hugo?resume=false}{A clear and concise beginner hugo tutorial}
	\item
	\href{https://youtu.be/yfoY53QXEnI}{CSS Crash Course for Absolute Beginners}
\end{itemize}

%%%%%%%%%%%%%%%%%%%%%%
\section{Design Patterns}
%%%%%%%%%%%%%%%%%%%%%%



\begin{quest}
\item Why use "design patterns"?
\begin{ans}
\begin{itemize}
	\item Design patterns let your write better code more quickly by providing a clearer picture of how to implement the design
	\item Design patterns encourage code reuse and accomodate change by supplying well-tested mechanisms for delegation, composition, and other non-inheritance based reuse techniques
	\item Design patterns encourage more legible and maintainable code
\end{itemize}
\end{ans}

\item Delegation? Composition?
\begin{ans}
\begin{itemize}
	\item delegation: a pattern where a given object provides an interface to a set of operations. However, the actual work for those operations is performed by one or more other objects.
	\item composition: Creating objects with other objects as members. Should be used when a "has-a" relationship appears.
\end{itemize}
\end{ans}

\item What are design patterns?
\begin{ans}
	content

\end{ans}

\item Which resources will you use to start learning about design patterns?
\begin{ans}
	GOF patterns (C++). Then, potentially Head First Design Patterns (Java)/
\end{ans}


\end{quest}

\subsection{References \& Further Reading}

\href{https://www.gofpatterns.com/design-patterns/module1/intro-design-patterns.php}{Introduction to Design Patterns Course}



%%%%%%%%%%%%%%%%%%%%%%
\section{Algorithms}
%%%%%%%%%%%%%%%%%%%%%%


\begin{quote}
	"4.5 years of learning programming and working as fullstack software engineer ... had interview with one of the FAANG companies this summer in Hong Kong but failed it due to the fact that I suck in DSA (Data Structures \& Algorithms)."
\end{quote}

\begin{quote}
	"I'm using to leetcode.com to learn data structures
	and algorithms since I got a rejection from FAANG after interviewing with them onsite."
\end{quote}

\href{https://blog.codechef.com/2020/07/24/the-role-of-data-structure-and-algorithms-in-programming/}{Role of DSA in Programming (July, 2020)}


