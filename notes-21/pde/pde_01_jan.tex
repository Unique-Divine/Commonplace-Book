\chapter{ODE Solving}

\section{Variation of Parameters}

I'll assume knowledge of how to solve homogeneous ODE such as the following ones. 

\begin{quest}
	\item $v = v(t)$. Solve $v'' + v = 0$. 
	\begin{ans}
		\[\boxed{v(t) = c_0 \cos(t) + c_1 \sin(t)}.\]
	\end{ans}

	\item $v = v(t)$. Solve $v'' - v = 0$.
	\begin{ans}
		\[ \boxed{ v(t) = c_0 \cosh(t) + c_1 \sinh(t)  }.\]
	\end{ans}

	\item How do we solve the more general $av''(t) + bv'(t) + c = f(t)$ for a general $f(t)$? 
	\begin{ans}
		Method of variation of parameters
	\end{ans}
\end{quest} 

Without any proof for why the method works, I'll go over how to use it. First, I'll need to talk about Cramer's rule and Wronskians. 

\subsection{Cramer's Rule}

Cramer's rule can be used to solve linear systems of equations that have a unique solution. We'll focus on systems of the following form since it will be relevant for variation of parameters:

\begin{gather*}
	X c' = \beta \\
	\begin{bmatrix}
		x_0 & x_1 \\
		x_0' & x_1' \\
	\end{bmatrix}  \begin{bmatrix}
		c'_0 \\ c'_1 
	\end{bmatrix} = \begin{bmatrix}
		0 \\ f
	\end{bmatrix},
\end{gather*}
where everything is a function of the same variable (let's call it $t$).

 \begin{note}
	$X$ is a called a ``Wronskian'' of $x_0$ and $x_1$. 
	\[ X = W(x_0, x_1) = \begin{bmatrix}
		x_0(t) & x_1(t) \\
		x_0'(t) & x_1'(t) \\
	\end{bmatrix} \]
\end{note}

This system has a unique solution if and only if $\det(X) \neq 0$. Let's suppose that's true. Since $\det(X) = x_0 x_1' - x_1 x_0' \neq 0$, we can easily solve for $c = X^{-1}\beta$. 
\begin{gather*}
	\begin{bmatrix}
		c'_0 \\ c'_1
	\end{bmatrix} = \frac{1}{x_0 x_1' - x_1 x_0'} 
	\begin{bmatrix}
		x_1' & -x_1 \\
		-x_0'& x_0 
	\end{bmatrix} 
	\begin{bmatrix}
		0 \\ f
	\end{bmatrix} = \frac{1}{\det(W(x_0, x_1))} \begin{bmatrix}
		-x_1 f \\
		x_0 f
	\end{bmatrix}
\end{gather*}

Equivalently, we could apply Cramer's Rule, which gives us the same answer from computing determinants. It's also requires less memorization. 

\[ c_0' = \frac{
	\begin{vmatrix}
	 0 & x_1 \\
	 f & x_1' \\
	\end{vmatrix}}
	{\det W(x_0, x_1)}, \;\;\;
	c_1' = \frac{
		\begin{vmatrix}
			x_0 & 0 \\
			x_0' & f \\
		\end{vmatrix}}
		{\det W(x_0, x_1)}.\]

Either way, we can solve for the desired coefficients since we know the value of their derivatives. 
\[ c_0(t) =  \int_0^t c_0'(t) \d t  +  d_0 \]
\[ c_1(t) =  \int_0^t c_1'(t) \d t  +  d_1 \]


\subsection{Method of undetermined coefficients}

The method of undetermined coefficients involves making educated guesses about the form of the particular solution to an ODE based on the form of non-homogeneous portion. 

A key pitfall of this method is that the form needed for the initial guess is not obvious. It's often better to use variation of parameters with Cramer's rule. 

\begin{quest}
	\item Find the particular solution to $y'' + 4y' + 3y = 3x$. 
	\begin{ans}
		The non-homogeneous component is a polynomial, so we assume particular solutions of polynomial form up to the order of the compoment, i.e. $Ax + B$.
		\begin{gather*}
			y_p = Ax + B \\
			\eval{\left( y'' + 4y' + 3y \right) }_{y_p} 
				= 0 + 4(A) + 3(Ax+ B) = 3x \\
			(3A)x + (4A + 3B) = 3x + 0 \\
			\therefore \;\; 4A + 3B = 0 \text{ and } 3A = 3.\\
			\therefore \;\;\; A = 1,\;\; B = -1. \\
		\therefore \;\; \boxed{ y_p = x - 1 } 
		\end{gather*}
	\end{ans}
	
	\item Find the homogeneous solutions to $y'' + 4y' + 3y = 3x$. 
	\begin{ans}
		Assume exponential solutions of the form $y_h(x) = e^{mx}$: 
		\begin{gather*}
			(m^2 + 4m + 3) e^{mx} = 0 \\
			m^2 + 4m + 3 = 0 \tag{$e^{mx} \neq 0$}\\
			(m + 3)(m + 1) = 0 \\
			m = \{-3, -1\}. \\ 
			\therefore \;\; \boxed{y_h = c_0 e^{-3x} + c_1 e^{-x}}
		\end{gather*}
	\end{ans}
\end{quest}

\chapter{Heat Equation}

\section{Equilibrium Temperature distribution (Haberman \S 1.4)}

The simple problem of heat flow: 

\begin{quest}
	\item \cloze If thermal coefficients are constant and there are no sources of thermal energy, then the temperature $u(x, t)$ in 1D rod $0\leq x \leq L$ satisfies \[ \partial_t u = k \partial_x^2 u .\]

	\item \cloze The above is known as the heat equation in 1D.

	\item What is the precise meaning of steady-state in relation to the heat equation?
	\begin{ans}
		If we say the boundary conditions at $x=0$ and $x=L$ are steady, that means they are independent of time. $\implies$ We define an equlibirum or steady-state solution to the heat equation is one that does not depend on time, i.e. $u(\vec{x}, t) = u(\vec{x})$.  
	\end{ans}

	\item Solve $\partial_x^2 u = 0$.
	\begin{ans}
		\begin{gather*}
			\partial_x^2 u = 0 
				\implies  \frac{\partial}{\partial x} (\partial_x u) = 0  \\
			\d(\partial_x u) = 0\cdot \d x \implies
				\int d (\partial_x u) = \int 0 \cdot \d x \\
			\therefore \partial_x u = C_0 \\ 
			\int \partial_x u \d x = \int C_0 \d x. \;\;
				\therefore\;\; \boxed{ \partial_x^2 u = C_0 x + C_1 } 
		\end{gather*}
	\end{ans}

	\item For equlibirium diffusion in a 1D rod with $x\in [0, L]$, what are the boundary conditions and  constraints?
	\begin{ans}
		Equilibrium $\implies u = u(\vec{x})$, $\iff u(0,t) = T_0 $ and $u(L, t) = T_1$. 
		
		Also, $\nabla^2 u = 0$.
	\end{ans}

	\item Determine the equilibrium temperature distribution for a 1-D rod ($x\in[0, L]$) with constant thermal properties with the following source and boundary conditions: 
	\[ Q = 0, \; u(0) = 0, \; u(L) = T .\]
	\begin{ans}
		\begin{align*}
			\text{PDE: }
				&\partial_t u = k\nabla^2 u + Q \tag{heat eq}\\
			\text{ODEs (equilibrium): }
				& k\nabla^2 u = 0.\;\; \partial_t u = 0 \\
				& \nabla^2 u = 0 \implies u = c_0 x + c_1 \\
			u(0) = 0 & 
				\implies c_1 = 0 \\
			u(L) = T &
				\implies c_0 = \frac{T}{L}. \\
			& \therefore \; \boxed{ u(x, t) = \frac{T}{L}x }.  
		\end{align*}
	\end{ans}

	\item 
\end{quest}




\subsection*{Lec. 3}

How do we find $B_n$?

Fourier's trick.
Multipl by

\[ f(x) = \sum\limits_{n=1}^\infty B_n \sin (\frac{n\pi}{L} x) \]

Orthogonality condition for sine 
\[ 
	\int\limits_0^L \sin(\npiell x) \sin(\mpiell x) \d x 
	= \begin{cases}
		\frac{L}{2} & m=n \\
		0 & m\neq n
	\end{cases}
\]

$ B_m  = \frac{2}{L} \int\limits_0^L f(x) \sin(\frac{m\pi}{L} x) \d x  $

\subsubsection*{Example - Diffusion Eq.}
Given: 
\begin{gather*}
	\partial_t  = k\partial_x^2 u \;\; t>0, x\in (0, L) \\
	u(0, t) = u(L, t) = 0 \\
	u(x, 0) = 1 \\
\end{gather*}

We just derived the solution to this general eq., which is 
\[ u(x, t) = \sum\limits_{n=1}^\infty B_n  e^{-k(\frac{n\pi}{L})^2  t} \sin (\frac{n\pi}{L}x) .\]

\[\begin{aligned}
	B_n &= \frac{2}{L} \int\limits_0^L f(x) \sin(\frac{n\pi}{L} x) \d x 
		= \frac{2}{L} \int\limits_0^L \sin(\frac{n\pi}{L} x) \d x  \\
		&= \frac{2}{L} \eval{
			\left( -\cos (\frac{n\pi}{L} x) \right)
		}
\end{aligned}
	\]


\subsection*{Lec. 4}

Recap:

We found that the 1D heat equation on a rod ($x\in(0,L)$), $\partial_t = k\partial_x^2 u $, subject to the following boundary conditions: 
\begin{gather*}
	u(0,t) = u(L,t) = 0 \\
	u(x, 0) = f(x) \\
\end{gather*}
has the solution 
\begin{gather*}
	u = \sum\limits_{n=1}^m B_n e^{-k \left(\npiell\right)^2 t} \sin(\npiell x) \\
	u(x, 0) = f(x) = \sum\limits_{n=1}^m B_n \sin(\npiell x) \\
	B_n = \frac{2}{L}\int\limits_0^L f(x) \sin(\mpiell x) \d x = 
		\begin{cases}
			0 &\text{ if $n$ is even} \\
			\frac{4}{n\pi} &\text{ if $n$ is odd}
		\end{cases} \\
  \therefore u(x, t) = \sum\limits_{n_{\text{odd}} \geq 1}^\infty \frac{4}{n\pi} e^{-k (\npiell)^2 t} \sin(\npiell x)
\end{gather*}



\paragraph*{\S 2.4: Heat conduction in a rod with insulated ends}

BCs: $\partial_x u ( 0, t) = \partial_x u (L, t) = 0 $

IC: 

Solutions of the form: $u(x, t) = A_0 + \sum\limits_{n=1}^\infty A_n e^{-k (\npiell)^2 t} \cos(\npiell x)$. 

Let's find the arbitrary coefficients $A_0$ and $A_n$ ($n\geq 1$). 
\begin{equation}
	u(x, 0) = f(x) = A_0 + \sum\limits_{n=1}^\infty A_n \cos (\npiell x)
\end{equation}

\begin{quest}
	\item Solve for $A_0$.
\end{quest}

Integrate and interchange the order of sum and integral. 
\[\begin{aligned}
	\int\limits_{0}^L f(x) \d x &
		=  \int\limits_0^L A_0 \d x + \sum\limits_{n=1}^\infty A_n \int\limits_0^L \cos (\npiell x) \d x \\ 
		&= A_0 L + \sum\limits_{n=1}^\infty A_n 
			\left( \frac{L}{n\pi} \eval{\sin(\npiell x)  }_0^L   \right)\\
		&= A_0 L 
\end{aligned} \]
\begin{equation}
	A_0 = \frac{1}{L} \int\limits_0^L f(x) \d x
\end{equation}

\begin{quest}
	\item Solve for $A_n$.
	\begin{ans}
		Fourier's Trick, i.e. multiply both sides by $\phi_m(x) = \alpha_1 \cos(\mpiell x)$ and integrate. 
		\[\begin{aligned}
			\intl_0^L f(x) \phi_m(x) \d x
				&= \suml_0^\infty A_n \intl_0^L \phi_n(x) \phi_m(x) \d x \\
				&= \suml_0^\infty A_n \intl_0^L \cos(\npiell x) \cos(\mpiell x) \d x
		\end{aligned}\]
		We know from the orthogonality relations of cosine that this integral is 0 everywhere except for $m = n$. Consequently, 
		\[\begin{aligned}
			&\intl_0^L f(x) \phi_m(x) \d x = A_m \intl_0^L \phi^2_m(x) \d x \\
			& A_m = 
				\frac{ \intl_0^L f(x) \phi_m(x) \d x }{ \intl_0^L \phi^2_m(x) \d x } = \frac{ \intl_0^L f(x) \phi_m(x) \d x}{ (\frac{L}{2})  }
				= \frac{2}{L} \intl_0^L f(x) \phi_m(x) \d x \\
		\end{aligned}\]
	\end{ans}

	\item 
\end{quest}
\begin{equation} 
	\therefore \;\; A_0 = \frac{1}{L} \int\limits_0^L f(x) \d x .
\end{equation}


\subsubsection*{Diffusion eq. in an insulated circular ring [Lec. 4, 1-21]}


BCs: Periodic boundary conditions
