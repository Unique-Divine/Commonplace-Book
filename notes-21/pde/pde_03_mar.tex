\chapter{Higher-Dimensioal PDEs}

Lecture 11, \S 7 of Haberman - post midterm

Wei Chung (TA) office hours now Thursday \@ 4pm

7.4 Statements of Theorems for Helmholtz Eq. 

\begin{enumerate}
  \item There may be eigenfunctions $\phi_\lambda$ for a single $\lambda$ that are lineraly independent.
  \item If $\lambda_1 \neq \lambda_2$, then $\phi_{\lambda_1}$ and $\phi_{\lambda_2}$ are orthogonal. 
\end{enumerate}
\begin{gather*}
  \laplacian \phi + \lambda \phi = 0 \text{ in region }R \\
  \text{ on boundary }\partial R \\
\end{gather*}

\begin{example}
Given: \begin{gather*}
  \laplacian \phi + \lambda \phi = 0 \\
  \phi = \phi(x, y) \\
  \text{Domain: }(x, y) \in A
\end{gather*}
Derive the Rayleigh quotient for $\lambda$. 

To get the RQ fo $\lambda$, multiply the Helmholtz eq. by $\phi$. 
\begin{align*}
  \to \quad
    & \phi\nabla^2\phi + \phi \lambda \phi = 0 \\
  \phi\lambda \phi &= -\phi\nabla^2\phi \\
  \phi\lambda \phi \d A &= -\phi\nabla^2\phi \d A \\
  \therefore\quad \lambda 
    &= \frac{-\iint_{\mathbb{R}^2} \phi\nabla^2\phi \d A}
      {\iint_{\mathbb{R}^2} |\phi|^2 \d A } \tag{1}
\end{align*}
Since $\laplacian\phi = \text{div(grad}\phi) = \nabla\cdot (\nabla\phi)$, this term can be expanded using the product rule for derivatives:
\begin{align*}
  &\frac{\d }{\d x} (fg) = \frac{\d f}{\d x} g + f\frac{\d g}{\d x} \\
  &\implies \nabla(fg) = \nabla f \cdot g + f \cdot \nabla g \\
  &\text{Let $f:= \phi$ and $g:= \nabla \phi$. Then, }\\
  &\nabla(fg) = \nabla(\phi\nabla\phi) 
    = \nabla\phi\cdot \nabla\phi + \phi\nabla\cdot (\nabla \phi) \\
  &\nabla(\phi\nabla\phi) = |\nabla \phi|^2 + \phi\laplacian\phi \\
  &\therefore\quad \phi\laplacian\phi 
    = \nabla\cdot (\phi\nabla\phi) - |\nabla\phi |^2 \tag{2}
\end{align*} 
From eq. 1 and 2, 
\begin{align*}
\lambda 
  &= \frac{-\iint_{A} \nabla \cdot(\phi\nabla\phi)\d A
    + \iint_{A} |\nabla \phi|^2 \d A}
  {\iint_{A} |\phi|^2 \d A } 
\end{align*}
By Green's theorem, $\iint_A \nabla \cdot \vec{F} \d A 
= \oint_{\partial A} \vec{F} \cdot \hat{n} \d \ell $, 
where $\d \ell$ denotes an infitesimal line element, and $\vec{F}$ is a field over the domain of $A$, and $\partial A$ denotes the boundary of $A$. Hence, 
\[ \boxed{ 
  \lambda 
  = \frac{-\iint_{\partial A} (\phi\nabla\phi)\cdot \hat{n} \d \ell 
    + \iint_{A} |\nabla \phi|^2 \d A}
  {\iint_{A} |\phi|^2 \d A }
} \]
\end{example}

\begin{example}
  Redo the previous example with the Sturm-Louiville eq. instead of the Helmholtz. 

  Given: 
  \begin{gather*}
    \nabla (p\nabla\phi) + q\phi + \lambda\sigma \phi = 0\\
    \phi = \phi(\vec{x}), p = p(\vec{x}), q = q(\vec{x}), 
      \sigma = \sigma(\vec{x}) \\
    \vec{x} := (x, y).  \\
    \vec{x} \in \text{ region $R$ with boundary curve $\partial R$}.  
  \end{gather*}

I'm going to call the region $A$ for area. 
\begin{align*}
&\implies \lambda \sigma \phi 
  = - \nabla\cdot(p\nabla \phi) - q\phi  \\  
& \lambda \sigma \phi \d A 
  = - \nabla\cdot(p\nabla \phi)\d A - q\phi \d A  \\
& \phi \lambda \sigma \phi \d A 
  = - \phi \nabla\cdot(p\nabla \phi)\d A - \phi q\phi \d A  \\
& \lambda \sigma |\phi|^2 \d A 
  = - \phi \nabla\cdot(p\nabla \phi)\d A - q|\phi|^2 \d A  \\
& \therefore \quad \lambda 
  = \frac{ - \iint_A \phi \nabla\cdot(p\nabla \phi)\d A 
    - \iint_A q|\phi|^2 \d A  }
    { \iint_A \sigma |\phi|^2 \d A} \tag{3} 
\end{align*}
\end{example}

\section{Dirac Delta}

he dirac delta $\delta(x - a)$ is a functional.
\begin{itemize}
  \item functional: A function that inputs a function and outputs a function. 
\end{itemize}

It's an "infinitely thin spike" at a single point and 0 everywhere else. 
\[ \delta(x - a) = \begin{cases}
  \infty  &,\quad  x = a  \\
  0 &, \quad \text{ else } 
\end{cases}\] This isn't the full description of the delta functional. It's most important property is that it integrates to 1. 
\begin{gather*}
  \int_{\mathcal{B}} f(x) \delta(x - x_0) \d x = f(x_0)
    \quad\text{ if }x_0\in\mathcal{B}  \tag{1} \\
  \therefore \quad 
    \int_{-\infty}^{+\infty} \delta (x - a) \d x = 1, 
    \quad a\in\mathbb{R} \\
\end{gather*}

Intuition: If you're within some boundary containing the spike and integrate over it, you get 1 because the spike is infinitely thin (and infinitely tall). If you're integrating over a function and the spike is at $x_0$, the integral will evaluate to the function's value at $x_0$ (Eq. 1).

\paragraph*{Other properties}
\begin{itemize}
  \item  Notice that you only get a spike at $\delta(0)$.  Thus, (Fact) \boxed{ \delta(x) = \delta(-x) } because $0 = -0$. A more general formula for this deals with transformations of the arguments
  \item (Fact) \boxed{\delta (k x) = \frac{1}{|k|} \delta(x)}. I'll state this property without proof. Note that it justifies my above claim: $\delta(-x) = \frac{1}{1} \delta (x) $.  
  
  Why would I use this? What if you have $\int_\mathbb{R} f(x)  \delta(3x - 2) \d x  $? It's a bit hard to interpret where the nonzero portion is because the answer is not to evaluate $f(3x-2)$. Instead, use 
  \begin{gather*}
    \delta(3x - 2) = \delta (3 (x - \frac{2}{3})) = \frac{1}{3} \delta(x - \frac{2}{3}) \\ 
    \therefore \quad 
    \int_\mathbb{R} f(x)  \delta(3x - 2) \d x  
    = \frac{1}{3}\int_\mathbb{R} f(x)  \delta(x - \frac{2}{3}) \d x = \frac{1}{3} f(\frac{2}{3})  
  \end{gather*}
\end{itemize} 

A common notation you'll sometimes see is $\delta_{nm} = \begin{cases}
  1 \quad &, n = m \\ 0 \quad &,  n \neq m 
\end{cases}$. This is called the Kronecker-delta. It's a similar concept but has a different meaning. 
\begin{gather*}
  \int_0^L \sin(\npiell x) \sin(\mpiell x) \d x = \frac{L}{2} \delta_{nm} \\
  \int_{-L}^{+L} \sin(\npiell x) \sin(\mpiell x) \d x = L \delta_{nm}
\end{gather*}


\chapter{Final Exam Review}

\paragraph*{Suggested Problems}
\begin{itemize}
  \item Chapter 7: 7.3.1, 7.3.4, 7.5.2, 7.5.7, 7.5.8 7.6.1, 7.6.2, 7.7.3, 7.7.9 7.9.1, 7.9.2
  \item Chapter 8: 8.2.2, 8.3.1, 8.4.1, 8.4.4 8.6.1, 8.6.2, 8.6.7
  \item Chapter 9: 9.3.6, 9.3.11, 9.3.14, 9.4.5, 9.4.7, 9.5.3, 9.5.4, 9.5.6
  \item Chapter 10: 10.3.8, 10.3.11, 10.4.4, 10.4.6, 10.4.7ab, 10.4.9
\end{itemize}



