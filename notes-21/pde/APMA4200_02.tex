\chapter{Laplace's Eq.}

\begin{itemize}
	\item Book \S 2.5
	\item Start: Lecture 4, 1-26 
\end{itemize}

\section{Rectangle - Laplace}

\begin{quest}
	\item Can we use separation of variables for $\nabla^2 u = 0$ with all nonhomogeneous BCs, and if so, under what conditions?
	\begin{ans}
		We can use separation of variables, however particular solutions that individually satisfy each nonhomogeneous BC must be added together with superposition. 
	\end{ans}

	\item Why is superposition of particular solutions justified in the context of Laplace's Eq.? 
	\begin{ans}
		Because Laplace's Eq. is a linear PDE ($\mathcal{L}(u) = 0 $).
	\end{ans}

	\item Let $\phi'' + \lambda \phi = 0$ with $\phi = \phi(x)$, $x\in [0, L]$, and $\phi(0) = \phi(L) = 0$. What are the eigenvalues and eigenfunctions?
	\begin{ans}
		The ODE solution is $\phi(x) = c_0 \sin (\sqrt{\lambda} x) + c_1 \cos (\sqrt{\lambda} x)$. Plug in the BCs. 
		\begin{gather*}
			\lambda_n = \left( \npiell \right)^2, \;\; n\in \mathbb{Z}^+ \\
			\phi_n(x) = \sin \left( \npiell x \right) 
		\end{gather*}
	\end{ans}
\end{quest}


Derivation for Laplace's Eq. in a rectangle:

$u(x, y) = u_{f0} + u_{f1} + u_{g0} + u_{g1}$

$u_{g0}(x, 0) = g_0(x), \;\;u_{g0}(x, H) = u_{g0}(0, y) = u_{g0}(L, y) = 0$.

Using superposition, 
\[ u(x, y) = \suml_{n=1}^\infty A_n \sin(\npiell x) \sinh (\npiell (y-H)) \].

To find the coefficients, $A_n$, we use orthogonality. 
\begin{gather*}
	g_0(x) := u(x, 0) = \suml_{n=1}^\infty A_n \sin (\npiell x) \sinh ( \npiell (-H) )  \\
	\implies \; A_n = \frac{1}{\sinh(\npiell (-H) )} \frac{2}{L} \intl_0^2 
		\sin(\npiell x) g_0(x) \d x
\end{gather*} 


\section{Circular Disk - Laplace}
$u = u(r, \theta), r\in(0, R), \theta\in(-\pi, \pi)$.
PDE: 
\begin{align*}
\nabla^2 u (r, \theta) = 0 
	= \frac{1}{r} \partial_r (r \partial_r u) + \frac{1}{r^2} \partial_\theta^2 u 
	= \frac{1}{r} \partial_r u + \partial_r^2 u + \frac{1}{r^2} \partial_\theta^2 u	 
\end{align*}

BCs: 
\begin{gather*}
	u(R, \theta) = \Theta(\theta) \tag{Outer edge B.C.} \\
	|u(0, \theta)| < \infty \tag{finite at origin} \\
	u(r, \pi) = u(r, -\pi) \tag{periodic I} \\
	\partial_\theta u (r, \pi) = \partial_\theta u(r, -\pi) \tag{periodic II}
\end{gather*}

Separate variables:
\[\begin{aligned}
	u(r, \theta) = G(r) \Theta(\theta) &\;\;\; \nabla^2 u = 0 \\
	0	&= \Theta \frac{1}{r} \partial_r ( r \partial_r G) + \frac{1}{r^2} G \partial_\theta^2 \Theta \\
	0	&= \frac{1}{Gr} \partial_r ( r \partial_r G) + \frac{1}{r^2 \Theta} \partial_\theta^2 \Theta = -\lambda + \lambda \\
	\therefore \; & \boxed{ \Theta'' + \lambda \Theta = 0 } \\
	& \boxed{ r\partial_r ( r\partial_r G) - \lambda G = 0 }
\end{aligned}\]

BCs in $\theta$:
\begin{gather*}
	\partial_\theta^2 \Theta + \lambda \Theta = 0. \;\; \Theta(\pi) = \Theta(-\pi). \;\; \Theta'(\pi) = \Theta'(-\pi) \\
	\therefore \;\; \boxed{\lambda_n = n^2, n\in \mathbb{N}. \;\; \phi_n = c_0 \sin(n\theta) + c_1 \cos (n \theta) } 
\end{gather*}

BCs in $r$: 
\begin{gather*}
	r\partial_r ( r\partial_r G) - \lambda G = 0 \\
	\implies r^2 G'' + rG' - \lambda G = 0 \\
\end{gather*}
Let $G(r) = r^p$. 
\begin{gather*}
	(p(p-1) + p - \lambda) r^p  = 0 \\
	p^2  = \lambda  = n^2. \; \implies p = \pm n \\
	\therefore \; G(r) = r^{\pm n}
\end{gather*}
In order to figure whether to take + or - $n$, impose the finite boundary condition:
\[ |G(0)| < \infty. \lim_{r\to 0} r^n = ; 0. \;\;\; \lim_{r\to 0} r^{-n} = \infty \]
Thus, $G(r) = r^n$.

So far, the general solution is 
\[ u(r, \theta) = G(r) \Theta(\theta) 
	= \suml_{n=0}^\infty A_n r^n \cos(n\theta) 
		+ \suml_{n=1}^\infty B_n r^n \sin(n\theta) \]

Last BC at $r=R$:
\begin{gather*}
	f(\theta) := u(R, \theta) = \suml_{n=0}^\infty A_n R^n \cos (n\theta) + \suml_{n=1}^\infty B_n R^n \sin(n\theta) \\
	A_0 
		= \frac{\int f(\theta) \d\theta}
			{\int\d\theta}  
		=  \frac{1}{2\pi} \intl_{-\pi}^\pi f(\theta) \d \theta \\
	A_n 
		= \frac{1}{\pi R^n} \intl_{-\pi}^\pi 
			f(\theta) \cos (n\theta) \d \theta \\
	B_n = \frac{1}{\pi R^n} \intl_{-\pi}^\pi ...
\end{gather*}

\chapter{Fourier Series}

Book \S 3

The Fourier series of $f(x)$ on $x\in[-L, L]$ is 
\begin{gather*}
	f \approx a_0 + \suml_{n=1}^\infty a_n \cos (\npiell x) + \suml_{n=1}^\infty b_n \sin(\npiell x) \\
	a_0 = \frac{1}{2L} \intl_{-L}^L f(x) \d x \\
	a_n = \frac{1}{L} \intl_{-L}^L f(x) \cos (\npiell x) \d x \\
	b_n = \frac{1}{L} \intl_{-L}^L f(x) \sin(\npiell x) \d x
\end{gather*}

Thm: $f$ piecewise smooth $\implies$ $f$ has finitely many corners and jumps. 2 results from this

\begin{example}
	\[ f(x) = \begin{cases}
		0 & x\in [-\pi, 0) \\
		2 & x\in [ 0, \pi]
	\end{cases}\]

	Fourier Coefficients
	\begin{align*}
		a_0 = \frac{1}{2\pi} \intl_{-\pi}^\pi f(x) \d x\\
		a_n = \frac{1}{\pi} \intl_{-\pi}^\pi f(x) \cos (nx) \d x = \frac{1}{\pi} \intl_0^\pi 2 \cos(nx) \d x \\
		b_n = \frac{1}{\pi} \intl_{-\pi}^\pi f(x) \sin(nx) \d x
	\end{align*}

	\begin{align*}
		f &\approx 1 + \suml_{\text{$n$ odd}} \frac{4}{n\pi} \sin(nx) \\
		&= 1 + \suml_{k=0}^\infty \frac{4}{(2k+1)\pi} \sin(n x) \tag{let $n = 2k+1$}
	\end{align*}

	At $x=0$, $f$ is not continuous. We have $f(0) = 2$. 
	
	At $x = \frac{\pi}{2}$:
	\begin{align*}
	2 = 1 + \frac{4}{\pi} \suml_{n=1}^\infty \frac{1}{2k+1} \sin((2k+1)\frac{\pi}{2}) \\
		\suml \frac{\sin((2k+1)\frac{\pi}{2})}{2k+1} = \frac{\pi}{2} \\ 
		\sin(\frac{(2k+1)\pi}{2}) ...
	\end{align*}
\end{example}

\section{Fourier Series of Odd/Even Functions}

An odd function integrated over symmetric interval is 0. Thus, the Fourier series of an odd function only has $\sin()$ terms. Similarly, the Fourier series of an even function has only $\cos()$ terms. 

\begin{quest}
	\item When does the Fourier Series have discontinuities?
	\begin{ans}
		\textbf{General Fourier Series:} A general Fourier Series for $f$ on $x\in (-L, L)$ is continuous as long as $f(x)$ is continuous and $f(-L) = f(L)$.
		
		\textbf{Cosine Series:} A coine series will be continuous as long as $f$ is continuous and $f(L) = f(-L)$ with $f$ extended in an even way

		\textbf{Sine Series:} A sine series will be continuous if $f$ is continuous, $f(0)=0$, and $f(L) = 0$. 
	\end{ans}
\end{quest}

Term by term time differentiation

When can one term by term differentiate a Fourier Series w.r.t. $x$? 


\section{Fourier Sine Series}

Recall that the temperature $u(x,t)$, in a 1-D rod $x\in(0,L)$ with $u(0,t)=u(L,t)=0$ satisfies 
\[ u(x,t) = \suml_{n=1}^\infty B_n \phi_n(x) e^{-\lambda_n kt},
	\quad \sqrt{\lambda_n} = \npiell , 
	\quad \phi_n(x) = \sin(\sqrt{\lambda_n} x). \]
The initial condition, $f(x) = \sum_{n} B_n \phi_n(x)$ is a series of sines, however our Fourier series defintion is defined over $x\in[-L, L]$, not $x\in[0, L]$. Also, $f(x)$ is not necessarily odd. In this sutation, we get the Fourier sine series by \textbf{extending  $f(x)$}. The odd extension of $f(x)$ is piecewise smooth as long as $f$ is piecewise smooth for $x\in[0, L]$. 

\begin{quest}
\item What is the Fourier sine series?
	\begin{ans}
	The Fourier sine series of $f(x)$ is the Fourier series of the odd extension of $f(x)$. 
	\end{ans}
\item $f(x) = x$ on $x\in[0, L]$. Derive the Fourier sine series of $f(x)$. 
	\begin{ans}
		Sine series representation: $x \sim \suml_{n=1}^\infty B_n \sin(\npiell x) , x\in[0, L] $. 
		The sine series is equal to $f$ on $x\in(-L, L)$ becuase there's no jump discontinuity at 0. If there was one, we could only say that $x = \suml_{n=1}^\infty B_n \sin(\npiell x), x\in(0, L)$.

		Thus, the Fourier sine series of $f(x)=x$ is 
		\begin{gather*}
			x = \suml_{n=1}^\infty B_n \sin(\npiell x), x\in(-L, L).  \\
			B_n = \frac{1}{(\frac{L}{2})} \intl_0^L 
				f(x) \sin(\npiell x) \d x 
				= \frac{2}{L} \intl_0^L x \sin(\npiell x) = \frac{2L}{n\pi} (-1)^{n+1}.   
		\end{gather*} 
		Use integration by parts to do the integral. 
	\end{ans}
\item Given $f(x) = \cos(\frac{\pi}{L} x)$ for $x\in[0, L]$. What is the Fourier sine series of $f(x)$? Only set up the problem. You need not solve. 
\begin{ans}
Fourier sine series representation: 
$\cos(\frac{\pi}{L} x) \sim \suml_{n=1}^\infty B_n \sin(\npiell x), x\in[0, L]$. 

This function has jump discontinuities at both 0 and $L$, so the equality between $f$ and its sine series holds only for $x\in(0, L)$ but not at $x=0$ or $x=L$. 
\begin{align*}
	&\therefore \quad \boxed{ \cos(\frac{\pi}{L} x) 
		= \suml_{n=1}^\infty B_n \sin(\npiell x), \quad x\in(0, L) } \\
	& \boxed{ B_n = \frac{1}{(\frac{L}{2})} 
		\intl_0^L \cos(\frac{\pi}{L} x)
		 \sin(\npiell x) \d x  } 
		= \begin{cases}
			0 &\text{$n$ odd} \\
			\frac{4n}{\pi(n^2-1)} &\text{$n$ even} 
		\end{cases}
\end{align*} 
\end{ans} 
\end{quest}

\section{Appendix A: Orthogonality relations for sine and cosine}

\subsection*{Asymmetric Boundaries}

\begin{align*}
&\int_0^L \sin(\npiell x) \sin(\mpiell x) \d x 
  =	\frac{L}{2} \delta_{mn}
	= \begin{cases}
		0  &\quad m \neq n \\
		L/2 &\quad m = n, 
	\end{cases} \\
&\int_0^L \cos(\npiell x) \cos(\npiell x) \d x 
	= \frac{L}{2} (1 + \delta_{n0}) \delta_{mn} 
	= \begin{cases}
		0 &\quad n\neq m \\
		L/2 &\quad n = m \neq 0 \\
		L &\quad n = m = 0 
	\end{cases}
\end{align*}

\subsection*{Symmetric Boundaries}
 
\begin{align*}
	&\int_{-L}^L \sin(\npiell x) \sin(\mpiell x) \d x 
	  = L \delta_{nm}
	  = \begin{cases}
			0  &\quad m \neq n \\
			L &\quad m = n \neq 0 , 
		\end{cases} \\
	&\int_{-L}^L \cos(\npiell x) \cos(\npiell x) \d x 
		= L (1 + \delta_{n0})\delta_{nm}	
		= \begin{cases}
			0 &\quad n\neq m \\
			L &\quad n = m \neq 0 \\
			2L &\quad n = m = 0  
		\end{cases} \\
  &\int_{-L}^L \sin(\npiell x)\cos(\mpiell x) \d x 
	 = 0 
\end{align*}



\chapter{Sturm-Louivile}  

\section{Exam 1 - \S 1-5}

\subsection{Exam 1 Info}
\begin{itemize}
	\item Today and Thursday not on exam
	\item Next Tuesday there's a review session that will basically be an office hours.
	\item There's a study guide on Courseworks. Ideally, you'll have done this before next Tuesday.
	\item Exam is next Thursday (Feb 25). No lecture. Exam is 90 minutes long and available from 9am EST Thursday to 9am EST on Friday. No typed solutions.
	\item There are free points available for completing an easy Gradescope ``quiz''. Must be done before 10am Monday Feb 22.
\end{itemize}





