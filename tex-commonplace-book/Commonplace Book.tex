\documentclass[11pt, fancy, bibstyle=apalike, cite=authoryear]{elegantbook}


% -------------------------------------------------------------
%				Packages
% -------------------------------------------------------------
\usepackage[ruled,vlined]{algorithm2e}

%\usepackage[style=authoryear,
%	bibstyle=authoryear,
%	citestyle=authoryear,
%	natbib=true,
%	hyperref=true,
%	backref=true,
%	abbreviate=true]{biblatex}
%	\addbibresource{master_references.bib}

\usepackage{wrapfig}
\usepackage{pdfpages} % include PDFs
\usepackage{xcolor, color, soul}
	% https://en.wikibooks.org/wiki/LaTeX/Colors

	% Highlight monospace text like markdown
	\definecolor{lightGray}{RGB}{230, 230, 230}
	\sethlcolor{lightGray}

\newcommand\code[2][black]
	{
	\textcolor{#1}{\hl{\texttt{#2}}}
	}

% Enable consolas for monofont
\usepackage{inconsolata}

% Code block options
\usepackage{listings}
% To add in a source file directly: \lstinputlisting[language=Python]{source.py}


\lstloadlanguages{C++,C,Python,bash,Java}

\lstnewenvironment{cpp}[0]{
	\lstset{
		language=C++,
	    basicstyle=\smaller\ttfamily,
	    keywordstyle=\color{cyan}\ttfamily,
	    stringstyle=\color{red}\ttfamily,
		showstringspaces=false,
	    commentstyle=\color{magenta}\ttfamily,
	    morecomment=[l][\color{magenta}]{\#}
		otherkeywords={=, +, [, ], (, ), \{, \}, *}
	}
}{}
\lstnewenvironment{java}[0]{
	\lstset{
		language=Java,
	    basicstyle=\smaller\ttfamily,
	    keywordstyle=\color{cyan}\ttfamily,
	    stringstyle=\color{red}\ttfamily,
		showstringspaces=false,
	    commentstyle=\color{magenta}\ttfamily,
	    morecomment=[l][\color{magenta}]{\#}
		otherkeywords={=, +, [, ], (, ), \{, \}, *}
	}
}{}
\lstnewenvironment{bash}[0]{
	\lstset{
		language=bash,
	    basicstyle=\smaller\ttfamily,
	    keywordstyle=\color{cyan}\ttfamily,
	    stringstyle=\color{red}\ttfamily,
		showstringspaces=false,
	    commentstyle=\color{magenta}\ttfamily,
	    morecomment=[l][\color{magenta}]{\#}
		otherkeywords={=, +, [, ], (, ), \{, \}, *}
	}
}{}
\lstnewenvironment{python}[0]{
	\lstset{
		language=Python,
	    basicstyle=\smaller\ttfamily,
	    keywordstyle=\color{cyan}\ttfamily,
	    stringstyle=\color{red}\ttfamily,
		showstringspaces=false,
	    commentstyle=\color{magenta}\ttfamily,
	    morecomment=[l][\color{magenta}]{\#}
		otherkeywords={=, +, [, ], (, ), \{, \}, *}
	}
}{}



% For hyperlinks
\usepackage{hyperref}
	\hypersetup{
		colorlinks=true,
		linkcolor=cyan,
		filecolor=magenta,
		urlcolor=cyan,
		citecolor=green
	}

%				Python Code Blocks
% -------------------------------------------------------------
%\usepackage{pythonhighlight}
\begin{comment}
% Python code block
\begin{python}
	def f(x):
	return x
\end{python}

% inline Python
\pyth{ <code>  }

% load external Python file from line 23 to line 50
\inputpython{python_file.py}{23}{50}
\end{comment}


% -------------------------------------------------------------
% 				Custom environments
% -------------------------------------------------------------
% \newenvironment{<env-name>}[<n-args>][<default>]{<begin-code>}{<end-code>}
\newenvironment{quest}{\begin{enumerate}[label=\textbf{Q: }]\bfseries}
	{\end{enumerate}}
\newenvironment{ans}{\par\normalfont}{}

\def\cloze{\textbf{(cloze) }}


% -------------------------------------------------------------
% 				Cover Information
% -------------------------------------------------------------

\title{Commonplace Book - Unique Divine}
% \subtitle{2020 - present}
\author{Unique Divine}
\institute{Columbia University}
\date{May, 2020 - present}
%\version{3.11}
\bioinfo{Bio}{\url{http://uniquedivine.xyz/}}
\extrainfo{Victory won\rq t come to us unless we go to it. }
\logo{lion_circular}
\cover{cover_sq.jpg}




% -------------------------------------------------------------

\begin{document}

\maketitle
\frontmatter
\tableofcontents


% 			Body

\mainmatter

\part{Artificial Intelligence \& Deep Learning}
\chapter{Deep Learning \& AI}

\section{}


\subsubsection*{Generative Adversarial Networks}

Read \cite{goodfellow2014generative}




\section{ML Finance Project}

\subsubsection*{\href{https://stackabuse.com/time-series-prediction-using-lstm-with-pytorch-in-python/}{example w/ multivariate time series in PyTorch}}


\begin{quest}
\item
	\cloze
	Neural networks can be constructed using the \pyth{torch.nn} package.

\item
	Import the package for constructing neural networks in PyTorch.
	\begin{ans}
		\pyth{import torch.nn as nn}
	\end{ans}

\item \cloze Seaborn comes with built-in datasets.

\item Load seaborn's flights dataset.
	\begin{ans}
		\pyth{flight_data = sns.load_dataset("flights")}
	\end{ans}

\item
	Why must time series data be scaled for sequence predictions?
	\begin{ans}
		When a network is fit on unscaled data, it is possible for large inputs to slow down the learning and convergence of your network and in some cases prevent the network from effectively learning your problem.
	\end{ans}

\item
	sklearn import for scaling data?
	\begin{ans}
		\pyth{from sklearn.preprocessing import MinMaxScaler}
	\end{ans}
\end{quest}





\begin{quote}
We know the field is fast moving. If the reader looking for more recent free reading resources, there are some good introductory/tutorial/survey papers on Arxiv; I happen to be compiling a list of them.
\end{quote}

 One of said review papers \cite{raghu2020survey}

\cite[hello]{raghu2020survey}

\section{Deep Learning for Genomic Risk Scores}

\begin{quotation}
``
A central aim of computational genomics is to identify variants (SNPs) in the genome which increase risks for diseases. Current analyses apply linear regression to identify SNPs with large associations, which are collected into a function called a Polygenic Risk Score (PRS) to predict disease for newly genotyped individuals. This project is broadly interested in whether we can improve performance of genomic risk scores using modern machine learning techniques.

A recent study assessed the disease prediction performance of neural networks in comparison to conventional PRSs, but did not find evidence of improvement. This project will explore whether neural networks can improve performance by incorporating gene expression data to the training process. Gene expression is often integrated with SNP data in Transcriptome-Wide Association Studies (TWAS), which bear some resemblance to neural network architectures with SNPs as input nodes, genes as intermediate nodes, and disease status as the output node. Modeling this process as a neural network however will require defining a more unconventional architecture in which a small subset of hidden nodes is anchored to observed values.

This project is designed for students with experience in machine learning topics and preferably with deep learning tools such as tensorflow or pytorch. Students should also be interested in applying machine learning and statistics to genomics applications.'' - Jie Yuan
\end{quotation}

\textbf{Terms to know}: Computational genomics, variants, single-nucleotide polymorphism (SNP), genome, Polygenic Risk Score (PRS), Transcriptome-Wide Association Studies (TWAS), gene(s), genomics


\subsection{Polygenic Risk Scores (paper) \cite{wray2010multi}}
\subsubsection*{Abstract (mining)}
\begin{description}
\item[recurrence risks] :
	In genetics, the likelihood that a hereditary trait or disorder present in one family member will occur again in other family members\footnote{\url{https://www.cancer.gov/publications/dictionaries/genetics-dictionary/def/recurrence-risk}}.

	``Evidence for genetic contribution to complex diseases is described by recurrence risks to relatives of diseased individuals.''

	This is distinguished from recurrence risk for cancer, which is the chance that a cancer that has been treated will recur.

\item[gene] :
	a sequence of DNA that codes for a specific peptide or RNA molecule; the physical and functional unit of heredity.

\item[locus] :
	the position of a gene on a chromosomes

\item[somatic cell] :
	any cell of the body except sperm and egg cells. A non-germline cell. any biological cell forming the body of an organism (except gametes).

	sôma (Ancient Greek): body

\item[genome] :
	An organism’s complete set of DNA, including all of its genes. Each genome contains all of the information needed to build and maintain that organism. In humans, a copy of the entire genome—more than 3 billion DNA base pairs—is contained in all cells that have a nucleus \footnote{\url{https://ghr.nlm.nih.gov/primer/hgp/genome}}.

	``genome-wide association''

\item[allosome] :
	(1) A sex chromosome such as the X and Y human sex chromosomes. (2) An atypical chromosome \footnote{\url{https://www.merriam-webster.com/medical/allosome}}.

	allo- (Greek): other, differnt

\item[autosome] :
	Any chromosome that is not a sex chromosome. The numbered chromosomes.

	auto (Greek): self, one's own, by oneself, of oneself

	-some, soma (Greek): body

\item[allele] :
	(genetics) One of a number of alternative forms of the same gene occupying a given position, or locus, on a chromosome.

	Borrowed from German Allel, shortened from English allelomorph. Ultimately from the Ancient Greek prefix allēl- from állos (“other”).

	``their effects and allele frequencies''

	allelomorph: another term for allele.

\item[risk loci] :

	``genome-wide association studies allow a description of the genetics of the same diseases in terms of risk loci...''

\item[haploid] :
	the quality of a cell or organism having a single set of chromosomes.

\item[diploid] :
	the quality of having two sets of chromosomes.

	``Sexually reproducing organisms are diploid'' (having two sets of chromosomes, one from each parent)

\item[eukaryotes] :
	Organisms whose cells have a nucleus enclosed within a nuclear envelope.

\item[gamete] :
	A mature sexual reproductive cell, as a sperm or egg, that unites with another cell to form a new organism. A haploid cell that fuses with another haploid cell during fertilization in organisms that sexually reproduce. A mature haploid male or female germ cell which is able to unite with another of the opposite sex in sexual reproduction to form a zygote.

	gamete (Ancient Greek): to marry

\item[zygote] :
	A eukaryotic cell formed by a fertilization event between two gametes.

	zygōtos (Greek): joined. yoked.

\item[monozygotic] :
	Monozygotic (MZ) or identical twins occur when a single egg is fertilized to form one zygote (hence, "monozygotic") which then divides into two separate embryos.



	``monozygotic twins''

\item[empirical] :

	``generate results more consistent with empirical estimates''

	\item[genetic variants]:
\end{description}
\begin{quest}
\item
	A human cell containing 22 autosomes and a Y chromosome is a sperm.


\end{quest}



\subsection{Neural Networks for Genomic Prediction (paper) \cite{pinto2019can}}

\subsection{Transcriptome Wide Association}



\part{Math \& Code}
% -----------------------------------------------------
\chapter{Java}
% -----------------------------------------------------

Let's dissect the following code block that sums two numbers.

\begin{java}
import java.util.*;
public class Solution {
    public static void main(String[] args) {
        Scanner in = new Scanner(System.in);
        int a, b;
        a = in.nextInt();
        b = in.nextInt();
        System.out.println(a + b);
    }
}
\end{java}

\begin{quest}
\item
	What does \texttt{String[] args} mean in the hello world program?

\begin{ans} \texttt{args} is an array of strings. \end{ans}
\end{quest}

\paragraph*{Arrays}
\begin{quest}
\item Declare an array of strings.
\begin{ans}
\begin{java}
String[] arr;
\end{java}
\end{ans}

\item Initialize an array of integers containing 1, and 2.
\begin{ans}
\begin{java}
int[] arr = {1, 2};
\end{java}
\end{ans}

\item Initialize an array of strings containing "BMW" and "Ford".

\begin{ans}
\begin{java}
String[] carBrands = {"BMW", "Ford"};
\end{java}
\end{ans}

\item Given an array,
\code{String[] carBrands = \{"BMW", "Ford"\}; }
print the first element.

\begin{ans}
\begin{java}
System.out.println(carBrands[0]);
\end{java}
\end{ans}
\end{quest}


\begin{quest}
\item
	\texttt{in.nextInt()} ?

\begin{ans}
content
\end{ans}

\end{quest}




It may be a good idea to go through all of this: \url{https://www.w3schools.com/java/default.asp}


\section{Gradle}

\url{https://youtu.be/aYu994I8Z6I?list=PLMxpKvJf0K0QUyvmkKZu7WpwTVzdePGTl}

\begin{itemize}
\item Gradle is an pen-source build atomation tool.
\item Gradle is the official android build tool; maintained by Android SDK Tools team
\item Gradle build scripts are written using Groovy
\end{itemize}


\paragraph*{WHy Gradle?}
\begin{itemize}
\item highly customizable and extensible
\item used for multiple languages
\item it's fast: reuses outputs from previous executions, processing only inputs that changed; executes tasks in parallel
\item higher performance than its peers (such as Maven)

\end{itemize}

\paragraph*{Gradle project tasks}




% -----------------------------------------------------
\chapter{Git}
% -----------------------------------------------------

Git (cookbook)

\section{Fundamental Concepts}

\subsection{Local branch vs. remote branch}
\begin{itemize}
\item \textbf{local branch}:  a branch only the local user can see. It exists only on your local machine.

	\begin{itemize}
	\item  Ex. Create local branch named "myNewBranch":

	\code{git branch myNewBranch }
	\end{itemize}

\item
\textbf{remote branch}: a branch on a remote location (in most cases `origin`). Local branches can be pushed to `origin` (a remote branch), where other users can track it.

	\begin{itemize}
	\item Ex. Push local branch, "myNewBranch", to the remote, "origin" so that a new branch named "myNewBranch" is created on the remote machine ("origin"):

	\code{git push -u origin myNewBranch}
	\end{itemize}


\item
\textbf{remote tracking branch}: A local copy of a remote branch.

When `myNewBranch` is pushed to `origin` using the command above, a remote tracking branch named `origin/myNewBranch` is created on your local machine.
\end{itemize}

\textbf{local tracking branch}: a local branch that is tracking another branch.

source(s): \href{https://stackoverflow.com/questions/16408300/what-are-the-differences-between-local-branch-local-tracking-branch-remote-bra}{SNce \& Brian Webster. stackoverflow.com}

\subsection{HEAD, master, and origin}

I highly recommend the book "Pro Git" by Scott Chacon. Take time and really read it, while exploring an actual git repo as you do.

\begin{itemize}
\item \textbf{HEAD}: the current commit your repo is on. Most of the time HEAD points to the latest commit in your current branch, but that doesn't have to be the case. HEAD really just means "what is my repo currently pointing at".

 In the event that the commit HEAD refers to is not the tip of any branch, this is called a "detached head".

\item
\textbf{master}: the name of the default branch that git creates for you when first creating a repo. In most cases, "master" means "the main branch". Most shops have everyone pushing to master, and master is considered the definitive view of the repo. But it's also common for release branches to be made off of master for releasing. Your local repo has its own master branch, that almost always follows the master of a remote repo.

\item
 \textbf{origin}: the default name that git gives to your main remote repo. Your box has its own repo, and you most likely push out to some remote repo that you and all your coworkers push to. That remote repo is almost always called origin, but it doesn't have to be.

\item
\code{HEAD} is an official notion in git. \code{HEAD} always has a well-defined meaning. \code{master} and \code{origin} are common names usually used in git, but they don't have to be.
\end{itemize}


source:  \href{https://stackoverflow.com/questions/8196544/what-are-the-git-concepts-of-head-master-origin}{HEAD, master, and origin. Matt Greer \& Jacqueline P. via stackoverflow.com}

\subsection{Large File Storage}

https://git-lfs.github.com/


\section{Git\S 2}

\subsection{SSH keys}

An SSH key is an alternative to username/password authorization on GitHub. This will allow you to bypass entering your username and password for future GitHub commands.


SSH keys come in pairs, a public key that gets shared with services like GitHub, and a private key that is stored only on your computer. If the keys match, you're granted access.

The cryptography behind SSH keys ensures that no one can reverse engineer your private key from the public one.

\url{https://jdblischak.github.io/2014-09-18-chicago/novice/git/05-sshkeys.html}

Generating a new SSH key: Follow \url{https://docs.github.com/en/free-pro-team@latest/github/authenticating-to-github/generating-a-new-ssh-key-and-adding-it-to-the-ssh-agent}





% -----------------------------------------------------
\chapter{C++}
% -----------------------------------------------------

C++ source code files end with a .cpp extension.

Hello world program: Run these and find out which one works.

\begin{cpp}
#include <iostream>

int main()
{
    std::cout << "Hello, world!";
    return 0;
}
\end{cpp}

Compiling and executing the C++ program:
\begin{enumerate}
\item
Step 1 is to install the gcc compiler.

\code{hi}


\item
Verify the install of g++ and gdb with
\code{whereis g++} and \code{whereis gdb}

To install gdb (linux or WSL), use \code{sudo apt-get install build-essential gdb}

\end{enumerate}


\paragraph*{\href{https://www.openmp.org//wp-content/uploads/openmp-examples-4.5.0.pdf}{Open MP}}

To import:
\begin{cpp}
#include <omp.h>
\end{cpp}

People use OpenMP for shared memory parallelization.


\subsection*{Header files}

C++ programs consist of more than just .cpp files. They also use **header files**, which can have a .h extension, .hpp extension, or even none at all.

\begin{quest}
\item What is a \texttt{.h} file?
\begin{ans}
header file
\end{ans}


\item What is the purpose of a header file?
\begin{ans}
Header files allow us to put declarations in one location and then import them wherever we need them. This can save a lot of typing in multi-file programs.
\end{ans}

\item What's contained in a \texttt{.h} file?
\begin{ans}
..
\end{ans}
\end{quest}


\subsection*{References \& Further Reading}
\begin{itemize}
\item
	\href{https://www.tutorialspoint.com/cprogramming/c_header_files.htm}{C header files}
\item
	\href{https://www.learncpp.com/cpp-tutorial/header-files/}{learncpp.com/.../header-files}
\end{itemize}

% -----------------------------------------------------

\chapter{Web development}
% -----------------------------------------------------


% -----------------------------------------------------
\section{Hugo Web Design}
% -----------------------------------------------------

Start Sep 16 (Mike Dane Tutorial Series)

\subsection{Intro to Hugo -\href{https://youtu.be/qtIqKaDlqXo}{(video)} }
\begin{itemize}
	\item
	Hugo is a static site generator.
	\item
	Static website generators allow you to compromise between writing a bunch of static html pages and using a heavy, and potentially expensive, content management system.
	\item
	Why Hugo? It's extremely fast.
	\item
	2 kinds of websites, dynamic and static. Dyanmic ex. Facebook. Facebook pages are dynamically generated for each user. For static websites, what you see is what you get.
	\item
	Static websites are notoriously harder to maintain b/c you lack some of the flexibility of things  on a dynamic site. Usually you can't use much conditional logic, functions, or variables.
	\item
	However, static pages are extremely fast.
	\item
	Hugo is great for a blog, portfolio website, etc.
	\item
	Hugo doesn't explicity require you to write a single line of HTML code.
	\item
	Flexibility | With that said, if you want to go in and change every little detail of the layout of the site, you cna do that. You can write as much of the HTML and have as much control as you'd like.
	\item
	Hugo is 100\% free and open-source.
\end{itemize}

\subsection{Intalling Hugo on Windows - \href{https://youtu.be/G7umPCU-8xc?list=PLLAZ4kZ9dFpOnyRlyS-liKL5ReHDcj4G3}{(video)} }
\begin{itemize}
	\item
	Mine's already installed. I'll skip this for now.
	\item
	a
	\item
	a
\end{itemize}

\subsection{Creating a new site - \href{https://youtu.be/sB0HLHjgQ7E?list=PLLAZ4kZ9dFpOnyRlyS-liKL5ReHDcj4G3}{(video)} }
\begin{itemize}
	\item skip, a bit too easy

\end{itemize}

\subsection{Installing \& Using Themes - \href{https://youtu.be/L34JL_3Jkyc?list=PLLAZ4kZ9dFpOnyRlyS-liKL5ReHDcj4G3}{(video)} }
\begin{itemize}
	\item my themes are already installed
\end{itemize}

\subsection{Creating \& Organizing Content - \href{https://youtu.be/0GZxidrlaRM?list=PLLAZ4kZ9dFpOnyRlyS-liKL5ReHDcj4G3}{(video)} }
\begin{itemize}
	\item
	Hugo has 2 types of content: single pages and list pages
	\item
	List content lists other content on the site. You can call this a list page.
	\item
	Individual blog posts are single pages.
	\item
	Your posts should not just be in the content directory. They should be in directories inside the content directory.
	\item
	A list page is automatically created for directories inside the content folder.  Hugo automatically does this. Note, this only occurs for directories at the ``root" level of the content directory. For example:  \texttt{content/post/} would generate a list page at \texttt{site.com/post/}, but \texttt{content/post/dir0} would not.
	\item
	If you want a list page to be generated for a dir that is not at the root level of the content dir, you have to create an \textbf{index filed}, \texttt{\_index.md}. For a convenient and efficient way to do this from the cmd, use \texttt{hugo new post/dir0/\_index.md} (above above example), then there will be a list page for dir0. 	Content can also be added to\texttt{\_index.md} and it should show up on the page.
	\item
	Additionally, for list pages that are aautomatically generate by hugo, you can edit the content by adding an index.md to those as well. Ex. \texttt{~content/post/\_index.md}.
\end{itemize}

\subsection{Front Matter- \href{https://youtu.be/Yh2xKRJGff4?list=PLLAZ4kZ9dFpOnyRlyS-liKL5ReHDcj4G3 }{(video)} }
\begin{itemize}
	\item
	Front matter in Hugo is what is commonly called meta data.
	\item
	Front matter is data about our content files.
	\item
	The metadata automatically generated by Hugo at the top of md files when using \texttt{hugo new } is front matter
	\item
	Front matter is stored in key-value pairs
	\item
	Front matter can be written in 3 different languages: YAML, TOML, and JSON
	\item
	The defualt lang for front matter in Hugo is YAML
	\item
	YAML - indicated by ``--'',
	\item
	TOML - indicated by "+++" and uses "=" instead of ":",
	\item
	JSON - indicated
	\item
	You can create your own custom front matter variables.
	\item
	Front matter is super powerful in its utility.
\end{itemize}


\subsection{Archetypes - \href{https://youtu.be/bcme8AzVh6o?list=PLLAZ4kZ9dFpOnyRlyS-liKL5ReHDcj4G3 }{(video)} }
\begin{itemize}
	\item
	How does the default front matter from using \texttt{hugo new ~.md} get selected? Short answer: archetypes
	\item
	An archetype is basically the default front matter template for when you create a new content file.
	\item
	Archetypes are modified under \texttt{static/themes/archetypes/default.md}
	\item
	Suppose your content dir has a subdirectory, \texttt{content/dir0}. If you wanted to create an archetype for the files in dir0, you'd simply create \texttt{dir0.md} inside the archetypes dir.
\end{itemize}

\subsection{Shortcodes - \href{https://youtu.be/2xkNJL4gJ9E?list=PLLAZ4kZ9dFpOnyRlyS-liKL5ReHDcj4G3 }{(video)} }
\begin{itemize}
	\item
	Shortcodes are predefined chunks of HTML that you can insert into your markdown files.
	\item
	Let's say you have a md file that you want to spice up by adding in some custom HTML. For instance, maybe you'd like to embed a YouTube video. Normally this would require lots of HTML that you'd have to paste it. Shortcodes can allow you to sidestep this. Hugo comes with a YouTube video shortcode predefined.
	\item
	General shortcode syntax \texttt{}
	\item
	Youtube shortcode | For a YouTube video with url, ``youtube.com/watch?v=random-text", the shortcode we'd use to embed would be \texttt{} because ``random-text'' is the id of the youtube video and the only parameter for that shortcode.
\end{itemize}

\subsection{Taxonomies - \href{https://youtu.be/pCPCQgqC8RA?list=PLLAZ4kZ9dFpOnyRlyS-liKL5ReHDcj4G3 }{(video)}}
\begin{itemize}
	\item
	Taxonomies in hugo are basically ways that you can logically group different pieces of content together in order to organize it in a more efficient way.
	\item
	Hugo provids 2 defualt taxonomies: tags \& categories
	\item
	All taxonomy information is declared in front matter.  In YAML, tags has the syntax \texttt{tags: ["tag0", "tag1", $\ldots$]}
\end{itemize}

\subsection{Templates - \href{https://youtu.be/gnJbPO-GFIw}{(video)}}
\begin{itemize}
	\item
	Templates here mostly refers to HTML templates. If you're not comfortable writing HTML, CSS, and coding for the web, templates might be a little bit above your head.
	\item
	A hugo theme is actually made up of hugo templates.
	\item
	Any template that you use in Hugo is going to be inside \texttt{themes/theme-name/layouts}. This is where all the templates live.
	\item
	\texttt{~/layouts/default} usually contains a default style for list and single pages by use of \texttt{list.html} and \texttt{single.html}.
\end{itemize}

\subsection{List Templates - \href{https://youtu.be/8b2YTSMdMps}{(video)}}
\begin{itemize}
	\item
	List templates give default HTML layout to list content files.
	\item

\end{itemize}

\subsection*{Resources}
\begin{itemize}
	\item
	\href{https://www.linkedin.com/learning/learning-static-site-building-with-hugo-2/build-a-static-site-with-hugo?resume=false}{A clear and concise beginner hugo tutorial}
	\item
	\href{https://youtu.be/yfoY53QXEnI}{CSS Crash Course for Absolute Beginners}
\end{itemize}

\chapter{Algorithms}

% -----------------------------------------------------
\section{Algorithms}
% -----------------------------------------------------


\begin{quote}
	"4.5 years of learning programming and working as fullstack software engineer ... had interview with one of the FAANG companies this summer in Hong Kong but failed it due to the fact that I suck in DSA (Data Structures \& Algorithms)."
\end{quote}

\begin{quote}
	"I'm using to leetcode.com to learn data structures
	and algorithms since I got a rejection from FAANG after interviewing with them onsite."
\end{quote}

\href{https://blog.codechef.com/2020/07/24/the-role-of-data-structure-and-algorithms-in-programming/}{Role of DSA in Programming (July, 2020)}





% -----------------------------------------------------
\section{Design Patterns}
% -----------------------------------------------------

\begin{quest}
\item Why use "design patterns"?
\begin{ans}
\begin{itemize}
	\item Design patterns let your write better code more quickly by providing a clearer picture of how to implement the design
	\item Design patterns encourage code reuse and accomodate change by supplying well-tested mechanisms for delegation, composition, and other non-inheritance based reuse techniques
	\item Design patterns encourage more legible and maintainable code
\end{itemize}
\end{ans}

\item Delegation? Composition?
\begin{ans}
\begin{itemize}
	\item delegation: a pattern where a given object provides an interface to a set of operations. However, the actual work for those operations is performed by one or more other objects.
	\item composition: Creating objects with other objects as members. Should be used when a "has-a" relationship appears.
\end{itemize}
\end{ans}

\item What are design patterns?
\begin{ans}
	content

\end{ans}

\item Which resources will you use to start learning about design patterns?
\begin{ans}
	GOF patterns (C++). Then, potentially Head First Design Patterns (Java)/
\end{ans}
\end{quest}

\subsection{References \& Further Reading}

\href{https://www.gofpatterns.com/design-patterns/module1/intro-design-patterns.php}{Introduction to Design Patterns Course}

\chapter{Miscellaneous}


\section{HTTP, REST, \& Spring Boot}

\begin{quest}
\item \cloze
REST has become the de-facto standard for building web services. Why? Easy to build, easy to consume.

\item
What is a micro service and why do we use them?
\begin{ans}
Microservice is an architecture that allows the developers to develop and deploy services independently. Each service running has its own process and this achieves the lightweight model to support business applications.

Microservices, a.k.a. the microservice architecture, structure an application as a collection of services that are
	\begin{itemize}
	\item Highly maintainable and testable
	\item Loosely coupled
	\item Independently deployable
	\item Organized around business capabilities
	\item Owned by a small team
	\end{itemize}
\end{ans}

\item What is Spring Boot?
\begin{ans}
An open source Java-based framework used to create microservices.
\end{ans}

\item Why use Spring Boot?
\begin{ans}
	\begin{itemize}
	\item manages REST endpoints
	\item flexible way to configure database transactions
	\item auto configured as opposed to manually configured
	\item eases dependency management
	\end{itemize}
\end{ans}

\item How does Spring Boot easy dependency management?
\begin{ans}
Spring Boot automatically configures your application based on the dependencies you have added to the project by using @EnableAutoConfiguration annotation. For example, if MySQL database is on your classpath, but you have not configured any database connection, then Spring Boot auto-configures an in-memory database.

The entry point of the spring boot application is the class contains @SpringBootApplication annotation and the main method.
\end{ans}
\end{quest}





\subsection{References \& Further Reading}

\href{https://spring.io/guides/tutorials/rest/}{Building REST services with Spring.}

\href{https://microservices.io/#:~:text=Microservices%20%2D%20also%20known%20as%20the,Organized%20around%20business%20capabilities}{What are microservices?}

\href{https://www.tutorialspoint.com/spring_boot/spring_boot_introduction.htm}{Spring Boot - Introduction}

\bibliography{references}

\end{document}

% -------------------------------------------------------------

% Template Recipes
\begin{comment}
% bold text
\textbf{}

% Inline code:
\lstinline{code}

% Note environment:
\begin{note}
	content
\end{note}

%Remark environment:
\begin{remark}
	On Overleaf, please use \hologo{XeLaTeX} to compile articles in Chinese and \hologo{pdfLaTeX} to compile articles in English.
\end{remark}

%Listing (code) environment.
\begin{lstlisting}
tlmgr update --self
tlmgr update --all
print("hello world")
\end{lstlisting}

%Assumption environment:
\begin{assumption}
	assumptions
\end{assumption}
\end{comment}











