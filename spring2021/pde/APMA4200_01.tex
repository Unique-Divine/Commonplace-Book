\chapter{Heat Equation}

\section{Lec. 1: Syllabus \& Logistics}
Prof will go over the expecations: expectations in terms of time commitment, workload, etc. We'll also talk about what you'll get out of this course.

\section{Equilibrium Temperature distribution (Haberman \S 1.4)}

The simple problem of heat flow: 

\begin{quest}
	\item \cloze If thermal coefficients are constant and there are no sources of thermal energy, then the temperature $u(x, t)$ in 1D rod $0\leq x \leq L$ satisfies \[ \partial_t u = k \partial_x^2 u .\]

	\item \cloze The above is known as the heat equation in 1D.

	\item What is the precise meaning of steady-state in relation to the heat equation?
	\begin{ans}
		If we say the boundary conditions at $x=0$ and $x=L$ are steady, that means they are independent of time. $\implies$ We define an equlibirum or steady-state solution to the heat equation is one that does not depend on time, i.e. $u(\vec{x}, t) = u(\vec{x})$.  
	\end{ans}

	\item Solve $\partial_x^2 u = 0$.
	\begin{ans}
		\begin{gather*}
			\partial_x^2 u = 0 
				\implies  \frac{\partial}{\partial x} (\partial_x u) = 0  \\
			\d(\partial_x u) = 0\cdot \d x \implies
				\int d (\partial_x u) = \int 0 \cdot \d x \\
			\therefore \partial_x u = C_0 \\ 
			\int \partial_x u \d x = \int C_0 \d x. \;\;
				\therefore\;\; \boxed{ \partial_x^2 u = C_0 x + C_1 } 
		\end{gather*}
	\end{ans}

	\item For equlibirium diffusion in a 1D rod with $x\in [0, L]$, what are the boundary conditions and  constraints?
	\begin{ans}
		Equilibrium $\implies u = u(\vec{x})$, $\iff u(0,t) = T_0 $ and $u(L, t) = T_1$. 
		
		Also, $\nabla^2 u = 0$.
	\end{ans}

	\item Determine the equilibrium temperature distribution for a 1-D rod ($x\in[0, L]$) with constant thermal properties with the following source and boundary conditions: 
	\[ Q = 0, \; u(0) = 0, \; u(L) = T .\]
	\begin{ans}
		\begin{align*}
			\text{PDE: }
				&\partial_t u = k\nabla^2 u + Q \tag{heat eq}\\
			\text{ODEs (equilibrium): }
				& k\nabla^2 u = 0.\;\; \partial_t u = 0 \\
				& \nabla^2 u = 0 \implies u = c_0 x + c_1 \\
			u(0) = 0 & 
				\implies c_1 = 0 \\
			u(L) = T &
				\implies c_0 = \frac{T}{L}. \\
			& \therefore \; \boxed{ u(x, t) = \frac{T}{L}x }.  
		\end{align*}
	\end{ans}

	\item 
\end{quest}


