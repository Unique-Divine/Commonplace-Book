
\chapter{Metalearning \& Doing}

\section{Ultralearning}


\textbf{Ultralearning}: It's about learning quickly and effectively.
- It's a strategy
- self-directed
- intense 
- rapid speed with great efficiency

\begin{quote}
	"Despite their idiosyncrasies, the ultralearners had a lot of shared traits. They usually worked alone, often toiling for months and years without much more than a blog entry to announce their efforts. Their interests tended toward obsession. They were aggressive about optimizing their strategies, fiercely debating the merits of esoteric concepts such as interleaving practice, leech thresholds, or keyword mnemonics. Above all, they cared about learning. Their motivation to learn pushed them to tackle intense projects, even if it often came at the sacrifice of credentials or conformity.”
\end{quote}

\paragraph*{How to become an ultralearner}
\begin{enumerate}
	\item \textbf{Metalearning}: First Draw a Map
	\item \textbf{Focus}: Sharpen Your Knife
	\item \textbf{Directness}: Go Straight Ahead
	\item \textbf{Drill}
	\item \textbf{Retrieval}: Test to Learn
	\item \textbf{Feedback}: Don't Dodge the Punches
	\item \textbf{Retention}: Don't Fill a Leaky Bucket
	\item \textbf{Intuition: Dig Deep Before Building Up} Develop your intuition through play and exploration of concepts and skills.
	\item \textbf{Experimentation}
\end{enumerate}

\subsection{\href{https://youtu.be/4xCiHppPfEs}{Core Message (YouTube)}}

A lot of what's taught in school is theory useful for PhD students.

An ultralearning project is about building skills, not just knowledge.

\subsubsection*{To Start Any Ultralearning Project}
\begin{enumerate}
	\item Make and follow a Metalearning map

	If I want to do $X$,
	\begin{itemize}
		\item what concepts do I need to understand?
		\item what facts do I need to memorize?
		\item what procedures do I need to practice?
	\end{itemize}


	Think hard about what you want to be able to do by the end of your ultralearning project. Then identify what skills will be critical to your success.

	\item Design and use practice drills
	\begin{itemize}
		\item Based on the info from metalearning, design drills.
		\item Practice drills should have quick feedback loops that target key areas, just like a bodybuilder would use dips to target the triceps.
	\end{itemize}

	\item Overlearn

	Go beyond the requirements of what is necessry to learn to further reinforce what IS necessary.

	To embrace overlearning, ask:
	\begin{itemize}
		\item what's my target performance, and
		\item what's the next level?
	\end{itemize}

	Commit to an even harder performance than the one you're training for.
\end{enumerate}

To recap:
1. Metalearn
2. Drill
3. Overlearn

\subsection{\href{https://www.scotthyoung.com/blog/2016/07/28/ultralearn-diy-1/}{How to Start Your Own Ultralearning Project (Part One)}}

Designing your own ultralearning project has three parts:

\begin{enumerate}
	\item Figuring out what you want to learn deeply, intensely and quickly.

	\item Choosing which format you want for your project.

	\item Preparing to start learning.
\end{enumerate}

\subsubsection*{Step 1: What Do You Want to Ultralearn?}

\subsubsection*{Step 2: Choose the Project Format}


\subsubsection*{Step 3: Preparing to Learn}

\section{Project Proposals}

\subsection{PyTorch | Deep Learning}

(August 25)

Structured Resources:
\begin{itemize}
	\item
	\href{http://www.deeplearningbook.org/}{Deep Learning Book}

	\item
	\href{https://www.youtube.com/results?search_query=sentdex+pytorch+pt.+3}{sentdex's PyTorch | Tutorial}

	\item
	\href{https://youtu.be/GIsg-ZUy0MY}{freeCodeCamp | Tutorial}
	\begin{itemize}
		\item (0:00:00) Introduction
		\item (0:03:25) PyTorch Basics \& Linear Regression
		\item (1:32:15) Image Classification with Logistic Regression
		\item (3:06:59) Training Deep Neural Networks on a GPU with PyTorch
		\item (4:44:51) Image Classification using Convolutional Neural Networks
		\item (6:35:11) Residual Networks, Data Augmentation and Regularization
		\item (8:12:08) Training Generative Adverserial Networks (GANs)
	\end{itemize}

	\item
	\href{https://youtu.be/pWrwyOsho5A}{Intro to PyTorch | Tutorial}

	\item
	\href{http://introtodeeplearning.com/}{MIT Deep Learning | Course}

	\item
	\href{https://deeplearning.mit.edu/}{MIT Deep Learning and AI | Lectures}

\end{itemize}

\paragraph*{Keywords:} Backpropagation, Hyperparameter, Feed-forward neural nets, Convolutional neural nets, Regularization

\subsubsection*{Meta Project:}
There's a list of tasks below. You can mix up the steps and bounce between tutorials or start working on the project before you've finished everything else, that's completely fine. The same goes for the reading. As long as everything gets done, consider this mini project a success.

\begin{itemize}
	\item
	Read through some of the deep learning book.

	\item
	Learn from sentdex's tutorial

	\item
	Complete a small project with PyTorch. Write it as a tutorial and share it with a few people.

	\item
	\href{https://youtu.be/GIsg-ZUy0MY}{freeCodeCamp | Tutorial}

	\item
	Gain high-level understanding of the keywords,

	enough that you could explain these concepts off the top of your head. Understand how they work and their significance.

	\item
	\href{https://youtu.be/pWrwyOsho5A}{Intro to PyTorch | Tutorial}
\end{itemize}

\paragraph*{Libraries | PyTorch}
ignite, fastai, skorch


\subsection{Blogging}

\subsection{Python DS Handbook}

It's Tuesday, August 18th and there are about two weeks until the semester starts.

I could definitely do with improving my Python data analysis fundamentals  before the semester starts. Both the Python DS Handbook and Wes Mickinney's book are awesome resources that I could use to up my skills. So here, I'm proposing a 6 day project to quickly and efficiently milk these resources.

\subsubsection*{Motivation:}
I've been trailing off on too many tangents. My recent project involving LSTMs sent me on an RNN binge that turned into a neural network binge that turned into a deep learning and AI binge. This isn't necessarily a bad thing. Some of that playful curiosity and interest was reignited in delving in this exploration. Heck, I even started reading academic papers for fun again.

Perhaps I'd lost sight of my true mission because I got too engrossed in the daily grind and tasks in front of me. Stopping to consider my interests didn't even cross my mind because lately I keep having to put out new fires. So, I took a step back. I wrote, I thought, and I've read  a lot in the past few days. I burned through dozens of books and articles, particularly on finance, investing, deep learning, metalearning, and productivity. And, I'm ready to go back and apply some more of what I've learned.

Read,

\subsubsection*{Proposal:}

You (I) have 6 full days to accomplish the follow, starting now:

\begin{itemize}
	\item
	Read (and highlight), note-take from Python DS Handbook.
	\begin{itemize}
		\item
		This is a pace of roughly 40 pages per day. Try to get through 100 a day.
	\end{itemize}
	\begin{enumerate}
		\item
		\textbf{Skim}. Search for interesting content you want to learn and relevant sections. Mark or record these sections for reading.
		\item
		\textbf{Read and Highlight} the sections you want to learn in more detail.
	\end{enumerate}
	\item
	Create content from the notes in the DS Handbook. This includes but is not limited to
	\begin{itemize}
		\item
		creating anki cards
		\item
		adding to your cookbook
	\end{itemize}

	\item
	Depending on how early you finish, you can decide whether you'd gain more from going through Wes's book or some other resource.
	\item
	\textbf{Also}, study 1-2 hours of Japanese to catch up on reps each day and exercise 30 minutes minimum.
\end{itemize}

\subsubsection*{Reflection}

\paragraph*{At a high level, tell me what happened:}

Surprise surprise. I didn't finish the reading goal, however this does not mean that the project was a failure. I believe I discovered pretty quickly in working on this reading project that I wasn't making the most time-efficient gains from taking a bottom-up approach to NumPy.

\paragraph*{What to change for future projects:}

I saw specific examples of NumPy concepts that would've helped me so much in my recent ML for Finance course. And the crazy thing was, I didn't need to know a lot of this info cold. I would've benefit just as much from having the familiarity of recognizing what utilities were even possible. Even just reading through the sections on broadcasting, masking, and universal functions would've helped me greatly.

If I could go back in time 3 months, I would read and highlight through this book cover to cover. I would take notes on some of the concepts and utilities available and potentially add them to a cookbook instead of making Anki card active recall questions of everything.

The time saved using the text as a reference instead of a tutorial would've helped me spend more time working on end-to-end projects, where my learning environment was stimulating, motivating, and had clear transferable value to potential employers and others around me.

\paragraph*{Key Takeaways:}
\begin{enumerate}
\item Read first.
\item Archive.

\end{enumerate}


\section{Getting Things Done (GTD)}

\begin{quote}
	``If your mind is empty, it is always ready for anything; it is open to everything''- Shunryu Suzuki
\end{quote}

\paragraph*{Basic workflow}
\begin{tabular}{ccc}
	1. Capture & 2. Clarify  &  3. Organize\\
	4. Reflect & 5. Engage &  \\
\end{tabular}

\subsection{GTD | Overview}
\paragraph*{Projects}
A project is any outcome for which you need more than one action step to achieve it. You often do not \emph{do} projects, you do actions which may be related to the completion of a project.

\paragraph*{Natural Planning of Projects} is a way to think of projects to create maximum value with minimal effort and time expenditure.
\begin{enumerate}
\item
	Why | Define the purpose, principles, and values of the project
\item
	What | Visualize the desired outcome
\item
	How | Brainstorm and organize ideas on how to accomplish your goal(s)
\item
	When / What now? | Identify your next immediate actions.
\end{enumerate}

\paragraph*{Capture}

You must capture all the things you consider incomplete in your universe; personal or professional, big or small, urgent or not.


\paragraph*{Clarify}
Clarifying is the act of
defining and clarifying, one by one, all the stuff in your inbox, until it is empty.

\begin{itemize}
\item
	(1) Discard a task if it is worthless, (2) incubate it in the Someday/Maybe list, or (3) keep it as Reference Material, if it is potentially useful information.

\item
	If an action takes less than two minutes, do it. If now is not the right time (be careful not to procrastinate), defer it to a specific time in the future.
\end{itemize}

paragraph*{Organize}
\begin{itemize}
\item
	To organize things that are actionable, you need a Projects List

\item
	The calendar is sacred territory. If you put something there, it is because it has to be done on the date indicated.
\end{itemize}

\paragraph*{Reflect}
The system cannot be static. If you want to be able to correctly choose your actions, you have to keep the system current. You need to
review the whole picture of your life and your work at regular intervals and at the appropriate levels.

\begin{itemize}
\item
	Every day, you will need to review your Calendar and Next Actions list organized by contexts.

\item
	Every week, you must review the remaining lists to keep your system clean, clear, complete and updated. The purpose of the Weekly
	Review is that your mind gets empty again.

\item
	Every so often you should review the big picture; clarifying your long-term goals and visions and principles that ultimately determine how you make your decisions.
\end{itemize}

\subsubsection*{Engage}

The purpose of all this management and workflow is to facilitate the decision of what you should eb doing at any time.

Your priorities are set by a hierachy of levels or perspective, from top down (1) your life purpose, (2) your vision of yourself in the future, (3) your medium-term goals, (4) your areas of responsibility, (5) your current projects and (6) the actions you need to do every day.

\subsection{GTD | Further Reading}

\href{https://facilethings.com/blog/en/science}{The Science Behind GTD}

\href{https://facilethings.com/blog/en/life-purpose}{How to find your life purpose}

\href{https://facilethings.com/blog/en/why-gtd-fails}{10 reasons why GTD will fail}

\href{https://facilethings.com/blog/en/weekly-review}{The Weekly Review, in detail}

\subsection{Review, in detail}

Review is a key component of GTD, to the point that \textbf{if you do not do it effectively, you are probably not following the methodology the right way}. It is a time we spend to ensure that we are always working on what is actually important, a way to be proactive and take control.

\subsubsection*{Get clear}

The first phase of review is to do clean-up, collecting everything and processing it. The result should be that everything is in the GTD system and the Inbox is empty. Clarify and assign dates to tasks.

\subsubsection*{Get current}

The second stage is to keep your system up to date. This is \textbf{absolutely necessary for you to fully trust your system}. Review the:
\begin{itemize}
\item
	\textbf{Inbox, past, and upcoming events} | Is there aything pending? Confirm that everything is done or update deadlines if necessary.
\item
	\textbf{Projects} | Evaluate the status of each project. Review the support material and add new actions if necessary. Make sure there is a next action for each active project.
\end{itemize}

\subsubsection*{Get creative}
You have everything under control. Now it is time to dig up old longings, to imagine, to dream, and to invent.
\begin{itemize}
\item \textbf{Someday/Maybe list} | Is it time to activate a project that was put off or abandoned? Are there things that no longer make sense?

\item \textbf{Imagine} | What are your big picture ambitions? How can you improve? What new ideas should you add to the system?

\end{itemize}

%%%%%%%%%%%%%%%%%%%%%%
%%%%%%%%%%%%%%%%%%%%%%
\chapter{How to}
%%%%%%%%%%%%%%%%%%%%%%
%%%%%%%%%%%%%%%%%%%%%%

\section{How to Read}
%%%%%%%%%%%%%%%%%%%%%%

\subsection{How to read a scientific paper}
 2017 scientific paper by Hubbard and Dunbar, about reading scientific papers.

 Hubbard, K. E., \& Dunbar, S. D. (2017). Perceptions of scientific research literature and strategies for reading papers depend on academic career stage. PloS one, 12(12), e0189753.

 image

\begin{enumerate}
\item Read the intro and background first, then the abstract (fast pass).
\item Read discussion (fast pass)
\item Attempt to interpret figures and define variables (highlight)
\item Re-read the other sections(highlight, slow pass)
\item Summarize the intro and background
\item Synthesize notes and ask questions. Record this somewhere (like a commonplace book)
\item Read methods, experiments, results, conclusion (fast pass). If necessary, go through a slow pass, highlighting and archiving the most salient points
\item \textbf{Don't forget} to take note of key references.
\item Attempt to solve all of your mysteries.
\end{enumerate}


%%%%%%%%%%%%%%%%%%%%%%
\section{Software Projects}
%%%%%%%%%%%%%%%%%%%%%%


\subsection{Portfolio Overview}

\paragraph*{Key factors for a Good Portfolio}
\begin{itemize}
	\item Completeness
	\item Functionality
	\item Readability
	\item Documentation/information
\end{itemize}

\subsection{Solve all of your mysteries}

The time it takes programmers to finish projects varies wildly. This section will go over why some people are spending 50+ hours on projects that should be done in 5.

\begin{itemize}
	\item
	Faster programmers can explan their code and strategy even if it  doesn't work.
	\item
	Faster programmers dig deep and really try to understand the functions, the classes, the API's that are at their disposal so that they can apply them to solve problems.
	\item
	Slower programmers are trying to look for a shortcut. They want to use stack overflow or some other site. This difference may not be noticeable on smaller beginner projects, but over time this difference becomes huge and will ultimately be crucial to one's long-term growth and flexibility.
	\item
	Never take code on faith

	\begin{itemize}
		\item
		If code is in your program, you should take time to learn about what it does
		\item
		Find some documentation. Read about the functions and classes used and DO NOT move forward until you understand each piece of code.
	\end{itemize}
	\item
	Read documentation
	\item
	Solve ALL of your mysteries
	\begin{itemize}
		\item
		You wrote a bunch of code, thought you knew what it did, and it's behaving mysteriously. A lot of times, people will try a couple of thing. They'll sort of randomly try stuff and even if the problem goes away and they don't really know why, they just move on. This is a bad idea.
		\item
		When you don't understand a bug, you are destined to repeat it. You need to understand why that bug occurred. Get out the documentation. Get a second pair of eyes, whatever you need to do. But, understand why this behavior is occurring.
	\end{itemize}
\end{itemize}
