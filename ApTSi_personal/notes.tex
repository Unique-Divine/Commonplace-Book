\part{Notes}

\chapter{The Beginning (2020 Oct. )}

\section{Onboarding (Sep 30, Meeting \# 1)}

Largely, what we want to get started with: \textbf{devloping the kafka test case integration}

\begin{itemize}
\item Get access to github once NDA is in
\item Our project management tool of choice is Trello: Kanban board
\end{itemize}



\paragraph*{Kafka:}
\begin{itemize}
\item tool used for streaming for high velocity flows
\item used typically in datalake
\item Integration will be from kafka to apache spark
\end{itemize}

\subsection*{Communication patterns:}
There are multiple communication patterns that you can have. The patterns differ based on the \textbf{velocity} of interactions.

\begin{itemize}
\item
	Synchronous deals with rest services/APIs.
	\begin{quote}
	"Eventually, you will 100\% be writing rest APIs on this project b/c our architecture's micro service based..."
	\end{quote}
	Rest services are synchronous by definition, which means there's a handshake. You send a request, somone sends back a response, but it's instantaneous. There's an immediate response that you need to send back.

\item
	Another common pattern for communication is \textbf{queued communication}. Queued communication is an example of \textbf{asynchronous communication}.

	\begin{quest}
	\item Why is it called asynchronous?
		\begin{ans}
			content
		\end{ans}
	\end{quest}

	callback: type of response

	\begin{itemize}
	\item 2 types of asynchronous communication: fire \& forget AND request-response driven (round track)
	\end{itemize}

	message ID, correlation ID: how to know which message

	queued communication: Consumer, $C$, drops things off in a queue. There's a queue handler on the side of the consumer. When $C$ sends something to the producer, $P$, this goes to the producer's queue handler.

	Cons: If one of the parties is slow, there can be blocking. The industry standard in response to this issue is "response codes".
\item
	TCPIP based communication
\end{itemize}

end ~ 10 minutes in




%\cite{goodfellow2016deep}
%
%\cite{Sing1503:Comment}
%
%\cite{goodfellow2014generative}
%
%\cite[hello]{raghu2020survey}
%
%\begin{quotation}
%\end{quotation}

